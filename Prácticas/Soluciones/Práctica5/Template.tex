\documentclass{article}

\usepackage[utf8]{inputenc}
\usepackage[spanish]{babel} % Agregar es-nodecimaldot si hay problemas con el separador de decimales y de miles
% \usepackage{noto-mono} % Tipografía "Noto Sans Mono" para monospace
\usepackage{amsmath, amsbsy, amssymb} % Agregar cuando se necesiten: amscd, amssymb, amsthm, latexsym
\usepackage{enumerate}
\usepackage{graphicx}
\usepackage{multicol}
\usepackage{svg}
\usepackage{tabularx}
\usepackage{algorithm}
\usepackage{algpseudocode}
\usepackage{caption}
\usepackage{etoolbox}

% Tamaño de hoja y márgenes
\usepackage[a4paper,top=1cm,bottom=2cm,left=1cm,right=1cm]{geometry}
\setlength{\parindent}{0em}
\setlength{\parskip}{1em}
\setlength{\intextsep}{1em}
\setlength{\abovedisplayskip}{1em}
\setlength{\belowdisplayskip}{1em}
\setlength{\abovedisplayshortskip}{1em}
\setlength{\belowdisplayshortskip}{1em}

% Numeración de las secciones y el índice
\setcounter{tocdepth}{1}
\renewcommand{\thesubsection}{\thesection.\alph{subsection}}

% Color de los links externos
\usepackage{hyperref}
\hypersetup{colorlinks=true, linkcolor=black, urlcolor=blue}

% Permite math mode adentro de listings (bloques de "código")
\usepackage{listings}
\lstset{
    inputencoding=utf8,
    extendedchars=\true,
    basicstyle=\ttfamily\small,
    mathescape,
    literate=%
        {á}{{\'a}}1
        {é}{{\'e}}1
        {í}{{\'i}}1
        {ó}{{\'o}}1
        {ú}{{\'u}}1
        {Á}{{\'A}}1
        {É}{{\'E}}1
        {Í}{{\'I}}1
        {Ó}{{\'O}}1
        {Ú}{{\'U}}1
        {ñ}{{\~n}}1
        {Ñ}{{\~N}}1
}

% Alias para íconos/símbolos
\usepackage{pifont}
\newcommand{\xmark}{\color{purple}\ding{54}}

% Otros alias
\newcommand{\xor}{\oplus}
\newcommand{\nor}{\downarrow}
\newcommand{\opsub}[2]{\ensuremath{#1_{\mathrm{#2}}}}
\newcommand{\yLuego}{\opsub{\land}{\scriptscriptstyle{L}}}
\newcommand{\oLuego}{\opsub{\lor}{\scriptscriptstyle{L}}}
\newcommand{\implicaLuego}{\opsub{\implies}{\scriptscriptstyle{L}}}
\renewcommand{\implies}{\Rightarrow}

% Columna "x" tiene fondo amarillo
\usepackage{colortbl}
\newcolumntype{x}{>{\columncolor[HTML]{FFF2CC}}c}

% Tipografía para los algoritmos
\AtBeginEnvironment{algorithmic}{\small}

% Saca sufijo "Algorithm X"
\captionsetup[algorithm]{labelformat=empty}

% Alias para algoritmos
\newcommand{\Complexity}[1]{\textbf{Complejidad}: #1}
\newcommand{\Pre}[1]{\textbf{Pre} $\equiv$ \{#1\}}
