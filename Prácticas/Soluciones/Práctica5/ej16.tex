\section{Ejercicio 16: raroSort}

\begin{algorithm}[H]
\caption{
    \textbf{RaroSort}(\textbf{in/out} A: arreglo(nat))
}
\begin{algorithmic}[1]
    \State m $\gets$ BuscarMax(A) \Comment{$O(n)$}
    \State d $\gets$ log(m) \Comment{Cantidad máxima de dígitos en base 2}
    \For{i $\gets$ 0 \textbf{to} d} \Comment{$O(d) = O(log(m))$}
        \State CountingSort(A, m, i) \Comment{$O(n+m)$, ordenamos por el i-ésimo dígito en base 2}
    \EndFor
\end{algorithmic}
\Complexity{$O((n+m) log(m))$} \\
Para revisar porque el enunciado pide $O(n log(m))$. No se si vale decir que $O(n+m) = O(n)$ en este contexto.
\end{algorithm}
