\section{Ejercicio 5: suma de potencias}

\begin{algorithm}[H]
\caption{
    \textbf{SumaDePotencias}(\textbf{in} A: arreglo(arreglo(nat)), \textbf{in} n: nat) $\to$ \textbf{out} res: arreglo(arreglo(nat))
}
\Pre{n = $2^k$ para algún k: nat $\geq$ 1}
\begin{algorithmic}[1]
    \If{n = 1}
        \State res $\gets$ A
    \Else
        \State res $\gets$ A + A * SumaDePotencias(A, n - 1)
    \EndIf
\end{algorithmic}
\Complexity{O(n)} \\
No usamos la función potencia. Este ejercicio es muy tramposo o lo hice mal.
\end{algorithm}
