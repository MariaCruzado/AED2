\section{Ejercicio 5: suma de potencias}

\begin{algorithm}[H]
\caption{
    \textbf{SumaDePotencias}(\textbf{in} A: arreglo(arreglo(nat)), \textbf{in} n: nat) $\to$ \textbf{out} res: arreglo(arreglo(nat))
}
\Pre{n = $2^k$ para algún k: nat $\geq$ 1}
\begin{algorithmic}[1]
    \If{n = 1}
        \State res $\gets$ A
    \Else
        \State B $\gets$ SumaDePotencias(A, n / 2)
        \State res $\gets$ Potencia(A, n / 2) * B + B
    \EndIf
\end{algorithmic}
\Complexity{ \\
    $T(n) = T(n/2) + cte$ \\
    Sea $a=1$, $b=2$, $f(n) = cte \in O(1)$ \\
    $f(n) = \Theta(n^{log_b(a)}) = \Theta(n^{log_2(1)}) = \Theta(n^0) = \Theta(1)$ \\
    $\implies T(n) = \Theta(n^{log_b(a)} log(n)) = \Theta(log(n))$ \\
    Asumimos que la función Potencia = O(1).
}
\end{algorithm}
