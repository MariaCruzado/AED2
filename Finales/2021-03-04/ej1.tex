\section*{Ejercicio 1}

\subsection*{a)}

\begin{itemize}
    \item \textbf{Los productos comprados son válidos}: Todos los productos de todas las compras son claves definidas del diccionario \lstinline{productos1} (los productos comprados se identifican por código). La vuelta no es necesaria porque es válido tener productos definidos que jamas fueron vendidos.
    \item \textbf{Las cantidades compradas son válidas}: Todas las cantidades de todas las compras deben ser mayor o igual a 1.
    \item \textbf{La cantidad total de ventas por producto se corresponde con las ventas registradas}: La cantidad total de ventas de un producto en particular es la suma de la cantidad vendida de ese producto entre todas las compras de todas las personas. A su vez debe valer la vuelta, todos los productos vendidos deben estar definidos en el diccionario \lstinline{ventas_por_prod}.
    \item \textbf{El mapeo entre el código del producto y su nombre es biyectivo}: El diccionario \lstinline{productos1} debe ser el inverso de \lstinline{productos2} (las claves pasan a ser significados). Necesitamos que todos los códigos mapeen a su nombre y que todos los nombres mapeen a su código.
\end{itemize}

\begin{lstlisting}
Rep: estr $\rightarrow$ bool
Rep(e) $\equiv$ true $\iff$ (
    ComprasVálidasYContabilizadas(e)
    $\land$ VentasPorProductoVálidas(e)
    $\land$ e.productos1 =obs InvertirDicc(e.productos2)
)

ComprasVálidasYContabilizadas: estr $\rightarrow$ bool
ComprasVálidasYContabilizadas(e) $\equiv$ true $\iff$ (
    ($\forall$p: persona)(p $\in$ claves(e.compras) $\implies$ (
        ($\forall$c: compra)(c $\in$ obtener(p, e.compras) $\implies$ (
            c.cant $\geq$ 1
            $\land$ def?(c.cod_prod, e.productos1)
            $\land$ def?(c.cod_prod, e.ventas_por_prod)
        ))
    ))
)

VentasPorProductoVálidas: estr $\rightarrow$ bool
VentasPorProductoVálidas(e) $\equiv$ true $\iff$ (
    ($\forall$cp: cod_prod)(cp $\in$ claves(e.ventas_por_prod) $\implies$ (
        obtener(cp, e.ventas_por_prod) = ContarVentas(cp, e.compras)
    ))
)

ContarVentas: cod_prod $\times$ dicc(persona $\rightarrow$ conj(compra)) $\rightarrow$ nat
Sea p = dameUno(claves(d))
Sea cs = obtener(p, d)
ContarVentas(cp, d) $\equiv$
    if vacío?(claves(d)) then
        0
    else
        ContarVentasAux(cp, cs) + ContarVentas(cp, borrar(p, d))
    fi

ContarVentasAux: cod_prod $\times$ conj(compra) $\rightarrow$ nat
Sea c = dameUno(cs)
ContarVentasAux(cp, cs) $\equiv$
    if vacío?(cs) then
        0
    else
        if c.cod_prod = cp then c.cant else 0 fi + ContarVentasAux(cp, sinUno(cs))
    fi

InvertirDicc: dicc(K $\rightarrow$ V) $\rightarrow$ dicc(V $\rightarrow$ K)
Sea k = dameUno(claves(d))
Sea v = obtener(k, d)
InvertirDicc(d) $\equiv$
    if vacío?(claves(d)) then
        vacío
    else
        definir(v, k, InvertirDicc(borrar(k, d)))
    fi
\end{lstlisting}

\subsection*{b)}

El diccionario \lstinline{productos1} mapea el código de producto a su nombre. Como el enunciado dice que la cantidad de productos no está acotada, podemos asumir que consecuentemente los códigos tampoco están acotados. Una primer opción podría ser representar este diccionario con un AVL donde las operaciones resultan $O(log(n))$ donde $n = |\text{productos1}|$.

Si logramos encontrar una cota $m$ para los códigos de producto tal que un porcentaje alto de ellos tienen un código $\leq m$ se podría representar el diccionario con 2 estructuras: un HashTable para los códigos $\leq m$ que permitiría realizar las operaciones clásicas en $O(1)$, y caso contrario el resto de los códigos se colocan en el AVL donde las operaciones resultan $O(log(n))$ donde $n$ es la cantidad de productos que tienen un código $> m$.

El diccionario \lstinline{productos2} mapea el nombre del producto a su código. En este caso podríamos representar el diccionario con un Trie ya que esta estructura es eficiente para operar con claves de tipo string. En el peor caso, las operaciones tendrían una complejidad de $O(log(k))$ donde $k$ es el nombre de producto que estamos buscando, insertando o borrando.
