\section*{Ejercicio 4}

¿Qué pasa si en hashing doble $h_1$ es una constante o si $h_2$ es una constante?

Hashing doble es una forma de direccionamiento abierto, en donde todas las claves van a parar adentro del HashTable $T$, obteniendo la posición de la siguiente forma:

$h(k, i) = (h_1(k) + i h_2(k)) \text{ mod } |T|$

\textbf{$h_1$ constante}

$h(k, i) = (c + i h_2(k)) \text{ mod } |T|$

El primer hasheo de cualquier clave siempre resulta $c$ pues $i=0$ inicialmente. Por lo tanto, una vez que hayamos insertado la primer clave, todas las futuras claves van a colisionar en el primer intento y tendremos que realizar un barrido para encontrar su posición. Dado una clave $k$ que colisionó, $h_2(k)$ resulta constante para esa clave en particular (las funciones de hash son determinísticas), y por lo tanto el barrido realiza saltos de $h_2(k)$ posiciones.

Puede haber aglomeración secundaria si $h_2$ produce el mismo valor para dos claves distintas, ya que en ese caso van a tener la misma secuencia de barrido a partir de la colisión inicial con $i=1$. Pero si $h_2$ produce valores distintos para 2 claves, ante una colisión ya no necesariamente va a ser necesario realizar la misma secuencia de barrido hasta encontrar una posición libre pues los saltos de los barridos de las 2 claves serían distintos (pero siguen siendo barridos lineales y existe la posibilidad de generar secuencias de barrido iguales dependiendo de la constante usada y el tamaño $|T|$).

\textbf{$h_2$ constante}

$h(k, i) = (h_1(k) + ic) \text{ mod } |T|$

La función de hash genera algo muy similar al barrido lineal en donde la posición inicial está determinada por la clave, y el barrido realiza saltos determinados por la constante $c = h_2$. A diferencia del caso anterior, acá la clave solo determina la posición inicial y el salto durante el barrido está fijo para todas las claves por igual.

Puede haber aglomeración primaria, ya que cualquier clave que colisiona con la aglomeración, va a seguir colisionando durante el barrido hasta llegar al ``final'' de la misma.

\textbf{Conclusión}

Si utilizamos una constante para $h_1$ o $h_2$ en una función de hashing doble, en esencia lo que sucede es que degradamos la función a un hashing simple con barridos lineales. En ambos casos sucede que para la mayoría de las constantes no vamos a poder asegurar que la función de hash produzca una permutación de todas las posiciones posibles de $T$.
