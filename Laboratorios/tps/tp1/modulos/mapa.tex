\subsection{Módulo Mapa}

\begin{Interfaz}
\textbf{se explica con}: \tadNombre{Mapa}

\textbf{géneros}: \TipoVariable{mapa}

{\textbf{\large Operaciones básicas}}

\InterfazFuncion{CrearMapa}{\In{hs}{conj(nat)}, \In{vs}{conj(nat)}}{mapa}
{$res \igobs crear(hs, vs)$}
[$O(copy(hs) + copy(vs))$]
[Crea un nuevo mapa con los ríos horizontales y verticales provistos.]

\InterfazFuncion{Horizontales}{\In{m}{mapa}}{conj(nat)}
{$res \igobs horizontales(m)$}
[$O(1)$]
[Devuelve los ríos horizontales, dando su ubicación en el eje vertical.]
[Genera aliasing, devuelve una referencia no modificable.]

\InterfazFuncion{Verticales}{\In{m}{mapa}}{conj(nat)}
{$res \igobs verticales(m)$}
[$O(1)$]
[Devuelve los ríos verticales, dando su ubicación en el eje horizontal.]
[Genera aliasing, devuelve una referencia no modificable.]

\end{Interfaz}

~

\begin{Representacion}

Un mapa contiene rios infinitos horizontales y verticales. Los ríos se
representan como conjuntos lineales de naturales que indican la posición en
los ejes de los ríos.

\begin{Estructura}{mapa}[estr]
    \begin{Tupla}[estr]
        \tupItem{horizontales}{conj(nat)}%
        \tupItem{verticales}{conj(nat)}%
    \end{Tupla}
\end{Estructura}

\Rep[estr]{true}

~

\Abs[estr]{mapa}[e]{m}{
    horizontales(m) $\igobs$ e.horizontales $\land$
    verticales(m) $\igobs$ e.verticales
}

\end{Representacion}

~

\begin{Algoritmos}

\begin{algorithm}[H]{\textbf{CrearMapa}(\In{hs}{conj(nat)}, \In{vs}{conj(nat)}) $\to$ \Out{res}{mapa}}
\begin{algorithmic}[1]
    \State res.horizontales $\gets$ hs \Comment $O(copy(hs))$
    \State res.verticales $\gets$ vs \Comment O($copy(vs))$
\end{algorithmic}
\textbf{Complejidad}: $O(copy(hs) + copy(vs))$
\end{algorithm}

\begin{algorithm}[H]{\textbf{Horizontales}(\In{e}{estr}) $\to$ \Out{res}{conj(nat)}}
\begin{algorithmic}[1]
    \State res $\gets$ e.horizontales \Comment $O(1)$
\end{algorithmic}
\textbf{Complejidad}: $O(1)$ \\
Pues se devuelven los ríos horizontales por referencia no modificable.
\end{algorithm}

\begin{algorithm}[H]{\textbf{Verticales}(\In{e}{estr}) $\to$ \Out{res}{conj(nat)}}
\begin{algorithmic}[1]
    \State res $\gets$ e.verticales \Comment $O(1)$
\end{algorithmic}
\textbf{Complejidad}: $O(1)$ \\
Pues se devuelven los ríos verticales por referencia no modificable.
\end{algorithm}

\end{Algoritmos}
